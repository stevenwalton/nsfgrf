% --------------- 12 POINT FONT -------------------------------
\documentclass[12pt]{article}
% --------------- 10 POINT FONT FOR CAPTIONS ------------------
\usepackage[font=footnotesize]{caption}
% --------------- NY TIMES FONT -------------------------------
\usepackage{times}
% --------------- 1 INCH MARGINS ------------------------------
\usepackage[margin=1in]{geometry}
% --------------- LINE SPACING --------------------------------
\usepackage{setspace}
\singlespacing
%\doublespacing
% --------------- SMALL SECTION TITLES ------------------------
\usepackage[tiny,compact]{titlesec}
% --------------- MATH PACKAGES -------------------------------
\usepackage{amsmath,amsthm,amssymb}
\begin{document}
% --------------- TITLE AND NAME ------------------------------
\begin{center}
\underline{\bf Personal Statement - Steven Walton}\\
\end{center}
From a young age I have always been interested in the sciences and engineering.
I grew up on sci-fi books and Star Trek. After I had destroyed a lot of things
in our house, attempting to figure out how they worked, my dad started handing
me broken things and challenging me to fix them before he bought replacements. I
have always had a strong desire to understand how everything works, from the
microscopic and mundane to the large and exciting areas. 

When I applied to university I was accepted as an aerospace engineer. I wanted
to explore the stars and thought that this would be the best way to make that
dream a reality. Before the semester was even over I had changed my degree to
Physics and working on a research project with the physics chair. I was more
interested in the toolbox that Physics gives to understand the world than
strictly building things. In experimental physics I found a nice balance between
doing hands applications and trying to understand things from a very basic
level. This passion continued to grow and I started taking on more projects from
other professors. 

My passion for the stars led to a professor asking me to help
with running the school's telescope and getting involved in outreach programs.
Our university would run science events a few times a year for students and
members of the community. We'd do events mainly around astronomical events like
solar eclipses, the Venus transit, and lunar eclipses. During the day events we
would set up a lot of different physics experiments that would get community
members interested in science and talk about the individual phenomena being
demonstrated. At night events we would bring out laser pointers and tell some of
the history and legends surrounding constellations. From these experiences I
developed a passion for teaching and became a TA for several of the physics
classes. I took a passion to trying to design lessons that would challenge
students but also keep them excited and engaged. 

and how you
are forced to think about complex problems. I've always liked to have a wide
breadth of knowledge and luckily was given the opportunities to do research for
several professors. In my junior year one professor gave me access to the optics
lab so that I could pursue my own projects. From that point on, on weekends I
could either be found hiking the Arizona mountains or in the optics lab trying
to replicate some experiment I read about. It was through my undergrad that I
learned how much I loved research. That it wasn't just understanding things, but
the reward of discovery. 

While in my undergrad I started programming a lot more. I had had a few
introductions to programming and computer science before, but had never had to
write anything complicated before. With some of the research the professors had
given me I had to start writing complicated code and gain a deeper understanding
of how programming works. 

After graduating I obtained a job in Tennessee to work at a rocket company.
There I was able to work on low level research in conjunction with NASA and the
University of Tennessee Knoxville. While there I won a Phase I SBIR contract and
continued the project into Phase II funding. 

\noindent\textbf{Intellectual Merit}
\noindent

\noindent\textbf{Broader Impacts}
\noindent

\end{document}
