% --------------- 12 POINT FONT -------------------------------
\documentclass[12pt]{article}
% --------------- 10 POINT FONT FOR CAPTIONS ------------------
\usepackage[font=footnotesize]{caption}
% --------------- NY TIMES FONT -------------------------------
\usepackage{times}
% --------------- 1 INCH MARGINS ------------------------------
\usepackage[margin=1in]{geometry}
% --------------- LINE SPACING --------------------------------
\usepackage{setspace}
\singlespacing
%\doublespacing
% --------------- SMALL SECTION TITLES ------------------------
\usepackage[tiny,compact]{titlesec}
% --------------- MATH PACKAGES -------------------------------
\usepackage{amsmath,amsthm,amssymb}
% --------------- Formatting ----------------------------------
%\setlength{\parskip}{\baselineskip}
\setlength{\parindent}{0pt}
\begin{document}
% --------------- TITLE AND NAME ------------------------------
\begin{center}
\underline{\bf Personal Statement - Steven Walton}\\
\end{center}
% Lead on societal change.
% More structure. Tell the reader what they should learn
I believe that our society faces significant challenges over the next century,
and am inspired to pursue a career in science because I believe science will
be the key to solving these challenges.
%
I began my education as a physicist, but have shifted to computer science since
I believe that machine learning will be essential to solving these societal challenges.
%
That said, I view my background as a strength, 
since my direction within computer science considers high-performance computing, 
computational science, and scientific visualization, and each of these are
made stronger with a physics background.
%
In this personal statement, I describe my experiences leading up to this proposal,
including my qualifications and why I believe I can succeed in answering the questions
outlined in this fellowship proposal.
%
In particular, I will describe my passion for science,
my desire to solve ethical dilemmas in Artificial Intelligence and bringing the
public to the conversation,
my passion for open science,
my desire to participate in science education,
my interdisciplinary skills,
why I believe I am an idea candidate for this work,
and how I believe I can have broader impacts on the public and scientific
community.
%In particular, I will describe my passion for science, 
%	my desire to participate in science education, 
%        my desire to help solve ethical dilemmas in artificial intelligence and 
%        bring the public to the conversation,
%	my experiences to date which include working in industry and obtaining funding, 
%         
%	...
%

\textbf{Passion for Science:} I grew up reading Science Fiction and watching
Star Trek. This had a lasting impact on how I view the world, showing how
science can change peoples' lives and the ethical decisions that need to be
made. Growing up with role models such as Isaac Asimov, Carl Sagan, and Captain
Kathryn Janeway, I have always been aware that while science can greatly advance
peoples' lives, there are new ethical dilemmas that arise. As another role model
might say: ``with great power comes great responsibility." As a researcher, I
want to act in a way that I believe would make these role models proud. I
believe that this requires me to be: part teacher, ensuring that my work is
accessible and available to others; part researcher, being creative and
performing the best work that I can; and part philosopher, questioning the
impacts and ethical dilemmas of my own research.

\textbf{Ethical Dilemmas in AI: }With the rise in popularity of Artificial
Intelligence and Machine Learning, it is becoming clear that ethical dilemmas
are growing and are frequently being unanswered, or ignored. Conversely, the
rise in AI research has helped push for more open science, where papers are
being published to locations such as ArXiv and code is being open sourced on
platforms such as GitHub. I believe that encouraging open science helps
everyone, from scientists to the public. I believe this openness will enable the
public to play a bigger role in both asking and answering the ethical dilemmas
involved with AI research. 

% Expand more on ethical dilemmas

\textbf{Passion for Open Science: }With open science, I believe the public can
be more informed with current research and help ask and answer ethical problems
that arise. When science is locked behind paywalls, it causes gate-keeping that
hinders all involved, especially the public and those in impoverished
neighborhoods and countries. Open research leads to higher reproducibility,
catching more mistakes, and allows the public to be involved in the process. I
also believe that when work is performed through public funding, that this work
should be accessible to those that funded it.

Having worked on several open source scientific projects as well as closed
source, I feel that open source projects are more readily adopted
by the scientific community. It also provides a direct line between developers and
users, which leads to higher quality software and a better understanding of
user needs. 

\textbf{Passion for Science Education: }I have always been passionate about
learning and trying to understand any problem that interested me. This lead me
to pursue an undergraduate degree in Space Physics. While I was able to buy all
the books that were required for my classes, I found that many times these were
not enough. Fortunately the material in undergraduate physics textbooks have not
change much in the last hundred years, enabling me to get old textbooks freely
or for very cheap. I found that sometimes the way one textbook was written would
be difficult for me to understand a topic, but that by going through several
textbooks I could find one that enabled the information to finally sink in. I
found that the more resources available to me, the better I performed. This made
me realize how important it is for experts to write on topics. Though there may
already be existing material that thoroughly explains a subject, sometimes it
requires a different voice for the reader to absorb the knowledge.

Like many other physicists, my colleagues and I admired Richard Feynman. When we
learned about the ``Feynman Technique," which says that one of the best ways to
learn is by teaching, we started to arrange our study groups so that one person
would lecture on a problem and the rest would ask as many questions as they
could. I found this exercise both challenging and exhilarating. Studying by
teaching was a great technique that helped me discover my passion for teaching.

Following through my passion for teaching, I became a Teaching Assistant for
several of the advanced physics labs. I had talked to my professor about my
enthusiasm for teaching and the possibility of becoming a professor after I went
through graduate school. To encourage me, he gave me the opportunity to lecture
several times in his place so that I could get the experience. This directly
lead to me becoming a Supplementary Instructor for a class in Computational
Methods, where I gave a mini lecture twice weekly. I found that this position
challenged me but had great rewards. I loved seeing students succeed and helping
them reach their potential. 

My undergraduate university frequently had outreach days, where we would discuss
different phenomena in physics with the public and the other colleges on the
university campus.

In my role, I had to design experiments that would both be exciting and could be
discussed in an elegant and understandable manner. Being a Space Physics
department, our outreach frequently revolved around astronomical events, where
we would provide the public and other students the ability to observe events
such as Lunar Eclipses, the Venus Transit, and several comets. I found great joy
in being a science communicator and want that role to always be part of my life. 




%% OLD
%From a young age I have always been interested in the sciences and engineering.
%I grew up on sci-fi books and Star Trek. After I had destroyed a lot of things
%in our house, attempting to figure out how they worked, my dad started handing
%me broken things and challenging me to fix them before he bought replacements. I
%have always had a strong desire to understand how everything works, from the
%microscopic and mundane to the large and exciting areas. 
%
%When I applied to university I was accepted as an aerospace engineer. I wanted
%to explore the stars and thought that this would be the best way to make that
%dream a reality. Before the semester was even over I had changed my degree to
%Physics and working on a research project with the physics chair. I was more
%interested in the toolbox that Physics gives to understand the world than
%strictly building things. In experimental physics I found a nice balance between
%doing hands applications and trying to understand things from a very basic
%level. This passion continued to grow and I started taking on more projects from
%other professors. 
%
%My passion for the stars led to a professor asking me to help
%with running the school's telescope and getting involved in outreach programs.
%Our university would run science events a few times a year for students and
%members of the community. We'd do events mainly around astronomical events like
%solar eclipses, the Venus transit, and lunar eclipses. During the day events we
%would set up a lot of different physics experiments that would get community
%members interested in science and talk about the individual phenomena being
%demonstrated. At night events we would bring out laser pointers and tell some of
%the history and legends surrounding constellations. From these experiences I
%developed a passion for teaching and became a TA for several of the physics
%classes. I took a passion to trying to design lessons that would challenge
%students but also keep them excited and engaged. I also found that his resulted
%in me having a deeper understanding of material and caused me to approach
%problems from different perspectives. 
%
%In undergrad I also got professors to give me extra access to different labs so
%that I could do research on my interests during my free time. Though the
%kindness of these professors I developed a deep passion for research. I felt
%that this access gave me the time to really dig into problems and concentrate on
%solving problems. The free reign gave me the ability to explore topics that I
%would not have been able to during classes and I believe helped develop my
%methods to solving complex problems. 
%
%While in my undergrad I started programming a lot more. I had had a few
%introductions to programming and computer science before, but had never had to
%write anything complicated before. With some of the research the professors had
%given me I had to start writing complicated code and gain a deeper understanding
%of how programming works. At this point I found programming interesting but it
%was a means to an end. 

\textbf{Interdisciplinary Skills: }After graduating I obtained a job in
Tennessee to work at a rocket company.  I was able to work on low-level research
in conjunction with NASA and the University of Tennessee, Knoxville (UTK). While
there, I wrote and won a Phase I NASA Small Business Technology Transfer (STTR)
proposal. My employers gave me the lead position on the project, where I was in
charge on planning, running the experiments, and handling the communication with
our University partner, UTK. I went on to obtain a Phase II funding and a
continuation of this work. A large part of this work was performing
computational simulations and then building and testing the resulting materials
that the simulations predicted would have the best performance. This was my
first time working with large and complex simulation code and I found that I
really enjoyed it. I found that programming really resonates with the way I
think and I found it exciting to build simulations that made real world
predictions and that I was making real contributions to science. 

This interest grew to a full on hobby, where I spent a lot of my time outside of
work learning new things about programming and computer science. I found that
there was a large amount of material on the internet that accelerated my
learning. Different blogs, papers, and videos helped me grasp concepts that I
had not learned through my undergraduate classes. I found many exciting topics,
including Artificial Intelligence and Machine Learning, and because of the open
nature of the research I was able to learn a lot about these subjects on my own.

After working for a few years, I decided that I wanted to continue down the path
of the intersection of computer science and the physical sciences. There were
new topics that I had been excited about and teaching myself but felt that this
would be more efficient if I could get a mentor that could help guide me to the
right problems and refine my abilities. I found the High Performance Computing
(HPC) visualization at the University of Oregon (UO) and applied for a Ph.D.
program, where I was accepted and have been a student for a year. 

During my studies at UO, I have been able to work on scientific visualization
projects and contribute to open sourced projects. I believe that my broad set of
skills has added value to our research group, as I can help bridge the
communication between domain scientists and the visualization experts. Through
my current advisor I have had the opportunity to work at several National Labs.
These internships gave me my first on-hands experience with HPC simulations and
the data analysis involved. Working on these projects I saw that there was need
for improvement, especially as there has been a push for larger scale computing.
With my interests in machine learning, physics, mathematics, and computation I
see ample opportunity to combine these topics and that my interdisciplinary
skills would enable me to research this intersection of Machine Learning and
computational analysis.
%While at UO I have been able to work on scientific visualization projects. I
%believe my broad skill set has has been added value to our research group as I
%can help bridge communication between the domain scientists and the
%visualization experts. Through my current advisor I have had the opportunity to
%work at Department of Energy (DoE) National Labs: Oak Ridge National Lab
%(ORNL) and Lawrence Livermore National Lab (LLNL). I got the first internship at
%ORNL the summer before I entered my Ph.D. program. This gave me my first real
%introduction into HPC and how the environment worked. During my second summer I
%was given the opportunity to work at LLNL and given the freedom to pursue my own
%interests. Here I was able to improve both my understanding in machine learning
%and visualizations in HPC. Over the summer I attempted to interpolate scientific
%data through machine learning, the project described within. There were some
%successes and failures in this, but I gained a better understanding of how to
%approach the problem and what tasks need to be solved within. Through my
%experiences here I believe I would be most happy working either in academia or
%at a lab, continuing research and helping mentor new generations of scientists. 

\textbf{Intellectual Merit}
%

With my background in physics and mathematics, coming over to the computer
science field I believe that I am in a unique position for a project such as
this. My strong math and physics background has already enabled me to help my
research group bridge gaps in communication between domain scientists and
visualization experts. It is not uncommon for miscommunication to happen between
different research domains, as researchers may be using similar words but in
different contexts. This has helped our group better understand what domain
scientists are looking for, but may have a difficult time communicating. Having
come from a background of performing simulations, it is easier to intuit what a
simulation scientist is trying to accomplish.

My strong background in mathematics has enabled me to quickly catch up with
modern machine learning techniques and understand the underlying statistical
properties of models. Being in a computer science environment has helped reduce
my so called ``Swiss Cheese" knowledge in computer science and HPC and allow me 
to progress towards expertise. 

\textbf{Broader Impacts}
%

I believe that one of the most important duties of a researcher is being able to
communicate their work. Scientists such as Carl Sagan and Neil deGrasse Tyson
have impacted my life and I believe their method of communication should be
emulated by the scientific community. I have greatly enjoyed my experiences
teaching classes and would like to become a professor in the future. I believe
that becoming a professor would allow me to follow through with my passion for
research and teaching. 

Additionally, I am passionate about open source software and blogging.
Personally, I have benefited greatly from both of these and blogging has enabled
me to learn a large breadth of subjects. In an effort to contribute back to the
community, which has helped me so much, I try to open source as much software as
I can and write in my blog about what I am learning. I believe that a NSF
fellowship puts me in a position that will encourage these passions. Not only
will it free up time to enable more blogging of my research, but through a
public funding I am not restricted from open sourcing my research and having the
research be in the open.

I believe that by open sourcing and blogging about my research that this can
more greatly enable reproducibility of results and push science to be more open.
It is my belief that science is performed more efficiently when more eyes are
able to see what is being done and more people are able to perform research.
With funding through the NSF, I am able to perform research without constraints
of visibility and can perform my research in the open.

%I not only have experience in teaching, but have enjoyed the opportunities that
%I have had to do so. An aspect that I greatly appreciate about my current
%research group is that there is a large focus on communication skills. Not only
%skills of how to communicate ideas between scientists in different domains, but
%to larger audiences. I believe that it is becoming increasingly important that
%all scientists become better communicators. Specifically I think we need to
%learn how to better explain our research to the general public and the
%importance of that work. Not only do I think it helps us get funding for our
%work, but helps get more people into science, and helps the world become more
%scientifically friendly. 
%
%I also think it is important for science to be done in the open, especially when
%it is publicly funded. In this effort I try to write about my own work and open
%source it through online media such as GitHub. I believe open source is both
%good for the scientific community as well as the public. It helps the scientific
%community reproduce and check work by other scientists. It also helps the public
%by making the science available to everyone. 

\end{document}

