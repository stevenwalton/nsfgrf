% --------------- 12 POINT FONT -------------------------------
\documentclass[12pt]{article}
% --------------- 10 POINT FONT FOR CAPTIONS ------------------
\usepackage[font=footnotesize]{caption}
% --------------- NY TIMES FONT -------------------------------
\usepackage{times}
% --------------- 1 INCH MARGINS ------------------------------
\usepackage[margin=1in]{geometry}
% --------------- LINE SPACING --------------------------------
\usepackage{setspace}
\singlespacing
%\doublespacing
% --------------- SMALL SECTION TITLES ------------------------
\usepackage[tiny,compact]{titlesec}
% --------------- MATH PACKAGES -------------------------------
\usepackage{amsmath,amsthm,amssymb}
% --------------- Formatting ----------------------------------
\setlength{\parskip}{\baselineskip}
\setlength{\parindent}{0pt}
\begin{document}
% --------------- TITLE AND NAME ------------------------------
\begin{center}
\underline{\bf Personal Statement - Steven Walton}\\
\end{center}
From a young age I have always been interested in the sciences and engineering.
I grew up on sci-fi books and Star Trek. After I had destroyed a lot of things
in our house, attempting to figure out how they worked, my dad started handing
me broken things and challenging me to fix them before he bought replacements. I
have always had a strong desire to understand how everything works, from the
microscopic and mundane to the large and exciting areas. 

When I applied to university I was accepted as an aerospace engineer. I wanted
to explore the stars and thought that this would be the best way to make that
dream a reality. Before the semester was even over I had changed my degree to
Physics and working on a research project with the physics chair. I was more
interested in the toolbox that Physics gives to understand the world than
strictly building things. In experimental physics I found a nice balance between
doing hands applications and trying to understand things from a very basic
level. This passion continued to grow and I started taking on more projects from
other professors. 

My passion for the stars led to a professor asking me to help
with running the school's telescope and getting involved in outreach programs.
Our university would run science events a few times a year for students and
members of the community. We'd do events mainly around astronomical events like
solar eclipses, the Venus transit, and lunar eclipses. During the day events we
would set up a lot of different physics experiments that would get community
members interested in science and talk about the individual phenomena being
demonstrated. At night events we would bring out laser pointers and tell some of
the history and legends surrounding constellations. From these experiences I
developed a passion for teaching and became a TA for several of the physics
classes. I took a passion to trying to design lessons that would challenge
students but also keep them excited and engaged. I also found that his resulted
in me having a deeper understanding of material and caused me to approach
problems from different perspectives. 

In undergrad I also got professors to give me extra access to different labs so
that I could do research on my interests during my free time. Though the
kindness of these professors I developed a deep passion for research. I felt
that this access gave me the time to really dig into problems and concentrate on
solving problems. The free reign gave me the ability to explore topics that I
would not have been able to during classes and I believe helped develop my
methods to solving complex problems. 

While in my undergrad I started programming a lot more. I had had a few
introductions to programming and computer science before, but had never had to
write anything complicated before. With some of the research the professors had
given me I had to start writing complicated code and gain a deeper understanding
of how programming works. At this point I found programming interesting but it
was a means to an end. 

After graduating I obtained a job in Tennessee to work at a rocket company.
There I was able to work on low level research in conjunction with NASA and the
University of Tennessee Knoxville. While there I wrote and won a Phase I NASA
STTR proposal. My employers gave me the lead position on the project, where I
was in charge on planning, running the experiments, and handling the
communication with our University partner (University of Tennessee Knoxville).
My success in this directly lead to Phase II funding and a continuation of this
work. A large part of this work was performing computational simulations and
then building and testing the resulting materials that the simulations predicted
would have the best performance. This was my first time working with large and
complex simulation code and I found that I really enjoyed it. I found that
programming really resonates with the way I think and I found it exciting to
build simulations that made real world predictions and that I was making real
contributions to science. 

With a few years of work under my belt I decided that I wanted to further
continue down the path of the intersection of computer science and the physical
sciences. There were new topics that I had been excited about and teaching
myself but felt that this would be more efficient if I could get a mentor that
could help guide me to the right problems and refine my abilities. I found the
HPC visualization at the University of Oregon (UO) and applied for a PhD program,
where I was accepted and have been a student for a year. 

While at UO I have been able to work on scientific visualization projects. I
believe my broad skill set has has been added value to our research group as I
can help bridge communication between the domain scientists and the
visualization experts. Through my current advisor I have had the opportunity to
work at two Department of Energy (DoE) National Labs: Oak Ridge National Lab
(ORNL) and Lawrence Livermore National Lab (LLNL). I got the first internship at
ORNL the summer before I entered my PhD program. This gave me my first real
introduction into HPC and how the environment worked. During my second summer I
was given the opportunity to work at LLNL and given the freedom to pursue my own
interests. Here I was able to improve both my understanding in machine learning
and visualizations in HPC. Over the summer I attempted to interpolate scientific
data through machine learning, the project described within. There were some
successes and failures in this, but I gained a better understanding of how to
approach the problem and what tasks need to be solved within. Through my
experiences here I believe I would be most happy working either in academia or
at a lab, continuing research and helping mentor new generations of scientists. 

\textbf{Intellectual Merit}
%
With my background in physics and mathematics coming over to the computer
science field I believe that I am in a unique position for a project such as
this. My strong math and physics background has already enabled me to help my
research group bridge gaps in communication between domain scientists and
visualization experts. It is not uncommon for miscommunication to happen between
different research domains, as researchers may be using similar words but in
different contexts. This has helped our group better understand what domain
scientists are looking for, but may have a difficult time communicating. Having
come from a background of performing simulations, it is easier to intuit what a
simulation scientist is trying to accomplish.

My strong background in mathematics has enabled me to quickly catch up with
modern machine learning techniques and understand the underlying statistical
properties of models. Being in a computer science environment has helped reduce
my so called ``Swiss Cheese" knowledge in computer science and HPC and allow me 
to progress towards expertise. 

\textbf{Broader Impacts}
%
I believe that one of the most important duties of a researcher is being able to
communicate their work. Scientists like Sagan and deGrasse Tyson have impacted
my life and I believe the how they communicate should be emulated. I have
greatly enjoyed my experiences teaching classes and would like to become a
professor in the future. I believe that becoming a professor would allow me to
follow through with my passion for research and teaching. 

Additionally, I am passionate about open source software and blogging.
Personally, I have benefited greatly from both of these and blogging has enabled
me to learn a large breadth of subjects. In an effort to contribute back to the
community, which has helped me so much, I try to open source as much software as
I can and write in my blog about what I am learning. I believe that a NSF
fellowship puts me in a position that will encourage these passions. Not only
will it free up time to enable more blogging of my research, but through a
public funding I am not restricted from open sourcing my research and having the
research be in the open.

I believe that by open sourcing and blogging about my research that this can
more greatly enable reproducibility of results and push science to be more open.
It is my belief that science is performed more efficiently when more eyes are
able to see what is being done and more people are able to perform research.
With funding through the NSF, I am able to perform research without constraints
of visibility and can perform my research in the open.

%I not only have experience in teaching, but have enjoyed the opportunities that
%I have had to do so. An aspect that I greatly appreciate about my current
%research group is that there is a large focus on communication skills. Not only
%skills of how to communicate ideas between scientists in different domains, but
%to larger audiences. I believe that it is becoming increasingly important that
%all scientists become better communicators. Specifically I think we need to
%learn how to better explain our research to the general public and the
%importance of that work. Not only do I think it helps us get funding for our
%work, but helps get more people into science, and helps the world become more
%scientifically friendly. 
%
%I also think it is important for science to be done in the open, especially when
%it is publicly funded. In this effort I try to write about my own work and open
%source it through online media such as GitHub. I believe open source is both
%good for the scientific community as well as the public. It helps the scientific
%community reproduce and check work by other scientists. It also helps the public
%by making the science available to everyone. 

\end{document}
